\hypertarget{introduction}{%
\section{Introduction}\label{introduction}}

Models have long been a critical part of the scientific method and thus
have been a significant focus of research within philosophy of science.
Models let scientists simplify, understand, and formalize intuitions,
ideas and concepts that come up in the course of research. Mathematical
models of physical phenomenal are essential parts of many major
developments in physics, including classical mechanics, general
relativity, quantum theory and many others. Experimentation on model
organisms, such as rats and mice, despite ethical debates, has become a
critical tool behind countless biomedical advances which resulted in new
vaccines and drugs. Atmospheric models have led to reasonably accurate,
if imperfect, weather forecasts, which can predict the paths of
hurricanes and save lives. An incorrect model of biological neurons has
led to artificial neural networks which routinely top the scoreboard of
image classification tests.

Because of the critical role that models play in science, there has been
extensive debate over their role in the scientific method and deep
questions about their epistemic status in the social sciences. At a high
level, it does seem fairly fishy that a scientist can devise a set of
rules, work out the consequences or those rules or run them as a
simulation and make a claim about the real world. However, how do we
square that with the central role they play in so many corners of the
research world? Are models merely about formalizing assumptions we
already make? Are all models wrong and only some useful? Are some
models, such as battle tested kinematic models fundamentally more true
than models from the social sciences which often are criticized for
being too simplified and idealized to be right? If so, how can we
rectify this issue to have better models in social science?

With all these big questions in mind, this thesis turns its attention to
the role of computational and mathematical modeling in philosophy. While
philosophy does not have the same deep connections with modeling as some
fields in the natural and social sciences, modeling has played a
significant role in several different fields of philosophy ranging from
social epistemology to social contract theory. While modeling has led to
highly influential works like the evolutionary account of the social
contract \autocite{skyrmsEvolutionSocialContract2014}, big questions
remain as to how exactly simulation should be deployed and what the
epistemic status of simulation-based studies is.

While many of the epistemic concerns about philosophical models are
shared with concerns over modeling and simulation in general, the nature
of the questions philosophers seek to answer with modeling can
accentuate these issues. For example, philosophers might want to use
simulations to help make normative claims about how we should act. In
this case, the simulation or model can't simply be benchmarked against
real-world measurements the way a weather model might be benchmarked
against observed temperatures. If the model is supposed to be a part of
a normative claim, we might expect the result to be different than
reality if we take it that reality might simply be imperfect.

The high level goal of this thesis is to search for a productive framing
for simulation in philosophy that has solid epistemic grounding, yet
allows for productive uses of simulation to help shed light on big
philosophical questions. While a sure answer to such a large question is
beyond the scope of a single work, this thesis seeks to point to a
possible productive direction and discuss what that framing might mean
for a controversial simulation-based philosophy paper.

This thesis explores the idea that epistemic framing of simulation-based
modeling in philosophy is more similar to that of animal-based models in
biology than to mathematical models of physical phenomena. Through this
lens, we see models as imperfectly replicating the actual mechanisms of
interest in the real world in a manner that allows for a high degree of
manipulation. The simulation becomes our lab rat that, when designed
carefully, can allow a researcher to manipulate the model in ways that
would otherwise be impractical or impossible on the actual mechanisms.
Because a simulation is mailable and facilitates manipulation, it
becomes an ideal starting point for exploratory and experimental
research as both of these modes are enabled by manipulation.

This framing brings up interesting questions about how these simulated
models might differ from physical models. It seems these experiments
rest on the fact that nature and evolution set the structure and
mechanism of the model. So then how could a simulation defined by a
researcher ever be an acceptable proxy for the real world? I'll argue
(and attempt to demonstrate) that by using detailed datasets about the
world, researchers can create quite convincing proxies for real world
mechanisms by using empirical data to define model mechanism.
Essentially, by defining part of a simulation with real measurements
rather than theoretical models, that simulation can more plausibly mimic
its target allowing the simulation to act more as lab rat than thought
experiment.

More concretely, this thesis wil begin with a discussion of models that
provides background on the salient issues about models in science, then
draws from mechanist and manipulationist accounts of science to motivate
my view of simulations as tools for exploration and experimentation.

Next, I will introduce Kevin Zollman's ``network epistemology'' project
which uses simulation to study the social epistemology of science
\autocite[\textcite{zollmanEpistemicBenefitTransient2009}]{zollmanNetworkEpistemologyCommunication2013}.
I'll frame some of the existing issues critics have flagged with the
simulations, most critically the parameter-sensitivity of the models
\autocite{rosenstockEpistemicNetworksLess2017a}. However, I'll argue
that this project can be framed better as experiments performed on a
simulation model of social interactions.

Finally, I'll take my framing of Zollman's model from \emph{The
Epistemic Benefit of Transient Diversity} and rework it to much more
convincingly represent social interactions by using a comprehensive
dataset of academic publishing and citation. The goal of this section is
to demonstrate that large datasets, when used to define model mechanism
structure, can produce more convincing baseline models. Essentially,
using lots of empirical data can provide a better lab rat than more
traditional graph structures could. This modification points to how a
more experimentally-oriented simulation project might have more clear
epistemic structure in accordance with my discussion of models. This
reworking required a significant engineering component in determining an
efficient way of generating convincing graph structure from a very large
citation dataset and ensuring the model itself could run efficiently on
these much larger graphs.

In sum, this thesis is about pointing toward a more sound and convincing
foundation for simulation-based modeling. I propose using empirical data
as a means of getting there and carry out an experiment which
demonstrates how that might work in practice. While reworking a single
study using this view is clearly inadequate to show thoroughly that this
experimental view is the best lens to view computational modeling, it
should hopefully demonstrate that there is untapped potential here.
Modeling could be a very effective tool to help sharpen intuitions and
try out ideas, so this thesis aims to show how modeling can be another
tool in the philosopher's research toolbox.
