\hypertarget{crafting-empirically-motivated-social-network-models}{%
\chapter{Crafting Empirically-Motivated Social Network
Models}\label{crafting-empirically-motivated-social-network-models}}

Given the mechanistic interpretation I've given of Zollman's model in
\emph{The Epistemic Benefit of Transient Diversity}
\autocite{zollmanEpistemicBenefitTransient2009} and my account of how
mechanistic models can be evaluated based on how well the model's
mechanism represents that of the target's mechanism, I wish to
demonstrate how viewing models in this way leads to a more robust
evaluation of a model. Because making the entire model more
empirically-motivated would be a difficult undertaking, especially given
the model includes models of poorly-understood phenomena like belief
representations in people, I choose to focus on the model mechanism
Zollman focuses on: the network structure of scientists.

This experimental section then has two distinct parts: developing a
realistic and defensible model of network structure following from
empirical data and verifying that new model by attempting to replicate
Zollman's modeling results on these larger, more complex graphs. My
primary goal with this experimental work is less to evaluate Zollman's
model, but rather to present empirically-defined mechanism as another
way to evaluate mechanistic models in terms of empirical distance, as
discussed in previous sections. Viewed in opposition to Rosenstock et
al.'s critique of Zollman's model on the grounds that the results don't
hold outside a specific parameter range, my experimental work seeks to
determine how the model performs on parameter values determined by
empirical data, rather than mathematical idealizations. In short, if the
model mechanism looks a lot like the target mechanism, then how does
Zollman's model perform?

\hypertarget{graph-terminology}{%
\section{Graph Terminology}\label{graph-terminology}}

First, I will define some background terminology that is critical to
understanding both Zollman's models and my extensions of them.

A graph \(G\), mathematically, is a tuple \(G = (N, E)\) where \(N\) is
a set of \emph{nodes} and \(E\) is a set of tuples of vertices where
\((n_i, n_j) \in E\) represents an \emph{edge} or connection between
node \(n_i \in N\) and node \(n_j \in N\). Nodes are often also referred
to as vertices, however, I adopt Zollman's use of ``node'' for clarity
here. This structure of nodes and relations between nodes shows up in
many fields such as model theory in modal logics, biological modeling of
complex systems, and social network modeling. Graphs are very adept at
capturing relational structure in the real world, which brings up an
important distinction between \emph{graph} and \emph{network}. When I
say graph, I refer to the mathematical structure given above. A graph
has no interpretation, it is merely some structure made of nodes and
edges. A \emph{network}, on the other hand, refers to a relational
structure in the real world which can be modeled as a graph. For
example, if we want to model the internet as a graph (simplistically),
nodes might be computers, and edges might be network links between them.

Graphs may be \emph{directed} or \emph{undirected}. In a directed graph,
an edge represents a one-way connection where the edge \((n_i, n_j)\)
denotes a connection from \(n_i\) to \(n_j\) but not \(n_j\) to \(n_i\).
In an undirected graph, all edges are bidirectional such that
\((n_i, n_j)\) means both \(n_i\) connects to \(n_j\) and \(n_j\)
connects to \(n_i\). In practice, an undirected graph can be realized in
a directed graph so long as the edge \((n_i, n_j) \in E\) implies the
presence of the edge \((n_j, n_i) \in E\).

Because graph structure can be so rich, there are far more ways to
quantify structure than can be detailed and deployed in a single paper.
Instead, I will introduce a few fundamental properties that are
essential to understanding Zollman's model here. A connected graph is
one in which there exists a path connecting any two nodes in the graph,
where a path is defined as a sequence of nodes \(n_1, \ldots, n_l\) such
that for each \(n_i\) in the path, there exists an edge
\((n_i, n_{i+1}) \in E\). In an undirected graph, the order of an edge
is ignored. For directed graphs, there are two senses of connectivity
\emph{weak} and \emph{strong}. Weak connections treat the directed graph
as undirected (ignoring the direction of edges) whereas strong
connections do not.

To illustrate these definitions, consider the following directed graph
\((N, E) = \left(\{ 1, 2, 3 \}, \{ (1, 2), (1, 3) \}\right)\). The graph
is weakly connected (and thus connected if undirected) because the node
\(1\) is connected to node 2 via \((1, 2)\) and to node 3 via
\((1, 3)\). Node 2 is connected to node \(1\) via \((1, 2)\) and to node
3 via \((1, 2)\) and \((1, 3)\). Node 3, similarly to node 2, is
connected to node \(1\) via \((1, 3)\) and to node 3 via \((1, 3)\) and
\((1, 2)\). Note that nodes 2 and 3 do no connect to any other node if
edges are instead directed. Thus, this graph is weakly connected, but
not strongly connected.

Finally, this definition of connectivity leads to the idea of
\emph{connected components} when combined with the idea of a
\emph{subgraph}. A \emph{subgraph} \(G' = (N', E')\) of a graph
\(G = (N, E)\) is formed by a subset of nodes \(N' \subseteq N\) and and
a subset of edges \(E' \subseteq E\) such that \(E'\) contains only
edges where both endpoints are in \(N'\). Now, a connected component is
a subgraph of \(G\) such that \(G'\) is connected. In addition, if the
\(G\) is directed and all nodes in \(G'\) is a strongly connected graph,
then \(G'\) is a strongly connected component of \(G\). Conversely, if
\(G'\) fits the definition of a weakly connected graph, then \(G'\) is a
weakly connected component.

Finally, there are several types of ideal graphs that are often
discussed and that Zollman uses as test cases in his paper. First, a
\emph{complete} graph is one in which every node is connected to every
other node in a graph. Second, a \emph{cycle} is a strongly connected
graph in which every node is connected to exactly two neighbors such
that all nodes are connected in one large ring. Finally a \emph{wheel}
is a strongly connected graph identical to a cycle, with the addition of
a single central node that is connected to every other node.

\hypertarget{zollmans-communication-networks}{%
\section{Zollman's Communication
Networks}\label{zollmans-communication-networks}}

While I will not rehash my framing of Zollman's model mechanistically in
this section, I will spend a bit of time focusing on the technicalities
of how the network structure part of the model operates. I'll begin with
a discussion of the intended target of Zollman's model to clarify what a
charitable interpretation of his work might be, before discussing how
this intention is represented in the actual model machinery.

Zollman discusses social networks, where he defined individuals as nodes
and edges to be the ``the communication of results from one to the
other'' \autocite[p.~25][]{zollmanEpistemicBenefitTransient2009}.
Furthermore, he posits that this relationship is symmetric, so if an
individual \(A\) can view individual \(B\)'s results, \(B\) can also
view \(A\)'s results. This definition can mean quite a few things in
practice and will prove too broad to apply directly to measurable
real-world behaviors. To demonstrate this, consider the following cases
where results are communicated:

\begin{enumerate}
\def\labelenumi{\arabic{enumi}.}
\tightlist
\item
  A scientist reads a published work, finds the results interesting, and
  eventually cites that work in her own work building off of or
  criticizing the published work.
\item
  A scientist reads a published journal article, finds the results
  uninteresting, and does not ever cite the work.
\item
  A scientist reads a news article about some research work, is
  influenced by the high-level ideas, but never reads underlying
  academic work, and thus does not cite it.
\item
  A scientist emails a friend in another lab for advice about starting a
  project and the friend reports that their results in the project area
  didn't look promising. Nothing is published, but info about results is
  transmitted.
\item
  A prominent scientist writes a blog or tweet reacting to a paper,
  influencing people's opinions of that paper without any published work
  to document it.
\item
  A scientist runs into another researcher at a conference and the two
  informally share ideas.
\end{enumerate}

Beyond these, there many other ways by which results may be communicated
within a scientific community with varying degrees of impact and
evidence associated with the transfer. Many of the more informal methods
of communication would be very hard to measure or quantify on a large
scale. Informal conversations and emails are rightly private and not
available for public analysis and social media is an emerging form of
communication for scientists which is not yet well-understood
\autocite{collinsHowAreScientists2016a}. Thus, these forms of
communication of results are difficult to measure and quantify.

Citations, on the other hand, formally recognize the specific results of
prior work which influenced the researcher. The APA's influential
publication manual formalizes this influence-based understanding of
citation, recommending that researchers ``cite the work of those
individuals whose ideas, theories, or research have directly influenced
your work'' \autocite{PublicationManualAmerican2010}. These citations
often take on a more strategic rhetorical purpose, explicitly building
on a paradigm of work started by another or criticizing that paradigm.
This argumentative conception of a citation's role in a scientific
community is characterized in greater depth by Bruno Latour in
\emph{Science in Action} as a part of his Actor-Network theory
\autocite{latourLiterature1987a}. However, for our purposes, we need not
wade into the subtleties here because no matter how a citation is
deployed in this sense, it trivially can be said to have influenced the
author. Even if the author is just cursorily familiar with the work and
is citing to criticize the approach, they display some awareness of the
results found. Citations don't always live up to this intent, as authors
may cite simply to help get by publication referees, and in some cases,
journals even force authors to cite certain articles, forming ``coercive
citations'' \autocite{wilhiteCoerciveCitationAcademic2012}. Despite
these degenerate cases, I argue it is fairly safe to assume that a
majority of authors who cite can be accurately described as receiving
and considering results from another party.

However, the reason to focus on citations over other forms of
communication is that citations do have norms which seem to be followed
to some extent and, critically, they are measurable empirically as a
result of the collection and systematization of academic metadata. While
I'll go into the practicalities of how in the next section, citation and
author data is easily available for large-scale analysis across nearly
every field of study. Because this data directly represents real
scientists citing the work of others, it has the potential to serve as a
convincing stand-in for what the real-world scientific communities might
look like.

So to create a more realistic account of network structure in science, I
plan to start from this data, rather than from the idealized graphs
Zollman proposed. The hope here is not that this data will prove a
perfect proxy for real communication, as no single data source captures
the full breadth of human communication, but as a much more
empirically-motivated structure than Zollman's rings, cycles and
complete graphs. However, to effectively deploy citation metadata to
create realistic communication structures, I must create a reasonable
rule for what constitutes evidence of communication and thus when to
draw an edge.

\hypertarget{constructing-realistic-graphs}{%
\section{Constructing Realistic
Graphs}\label{constructing-realistic-graphs}}

I divide this section into three distinct parts. First, I detail why I
selected the Microsoft Academic Graph (MAG) and what specific data it
contains. Then, I briefly detail my strategy for processing the large
dataset and finally conclude with a more formal definition of how I
define communication in terms of the specific metadata present in the
MAG.

\hypertarget{comprehensive-metadata}{%
\subsection{Comprehensive Metadata}\label{comprehensive-metadata}}

To infer communication from an academic social network, we first begin
with a comprehensive dataset of academic publishing metadata. By
publishing metadata, I refer to all the data associated with a scholarly
journal article, conference paper, or book except for the actual text of
the work itself. A good way to think about such metadata is everything
found on a ``works cited'' page: the title, a list of authors, the date
of publication, the journal, etc. However, most datasets also go beyond
this metadata and also include a set of citation links to other papers,
a set of fields of study as assigned by publishers as well as links to
other works that reference that paper.

Luckily several comprehensive sources of this metadata exist. Clarivate
Analytics compiles ``Web of Science'' (WoS), a proprietary dataset that
has vetted and comprehensive scientific metadata spanning from 1900 to
the present and including over 200 million entries. However, I decided
against using this dataset, which is commonly used in bibliometric
papers which analyze citation metadata, due to the closed nature of the
dataset which create barriers for researchers. There is also a young
project, OpenCitations, which seeks to build a citation data repository
that is entirely open with vetted data from publishers, however, the
project is still young and coverage sparse covering fewer than half a
million works
\autocite{peroniOpenCitationsInfrastructureOrganization2020}. For this
project, I chose to use the MAG
\autocite{sinhaOverviewMicrosoftAcademic2015b,wangReviewMicrosoftAcademic2019}
which leverages Bing's web indexing service much like Google Scholar
leverages Google's search infrastructure to generate a comprehensive
account of academic metadata, covering over 170 million entities.
However, unlike WoS, the data is available under a permissible license
(the Open Data Commons Attribution License) at no cost to the
researcher. This drove me to select the MAG over the others because it
presents the fewest barriers to follow on work and allows me to make my
results freely available.

\hypertarget{processing-a-large-graph}{%
\subsection{Processing a Large Graph}\label{processing-a-large-graph}}

The specific data in the MAG is evolving constantly, however, I use a
snapshot taken in October of 2018 for all my analysis. My MAG snapshot
is a ``multi-graph'' or a graph with different node types for authors
(\texttt{Author}), papers (\texttt{Paper}), fields of study
(\texttt{FieldOfStudy}) and journals (\texttt{Journals}). Furthermore,
there are directed edges for citations which connect papers to papers by
their listed citations (\texttt{REFERENCES}), from papers to their
authors (\texttt{AUTHORED\_BY}), from papers to their fields of study
(\texttt{IN\_FIELD}), and from subfields to their parent fields
(\texttt{PARENT}). These relations are formatted as a series of very
large CSV files (10-100GB each) and are not easily searchable in their
raw form as a result. Such large files do not fit in RAM on any
affordable machine, meaning it is not feasible to use the graph in its
entirety for simulations. Furthermore, understanding and visualizing the
results of a simulation of such a large size would be a difficult
undertaking.

To make this data useable, I decided to use the popular graph database
neo4j \autocite{Neo4j} to quickly query for smaller chunks of the
overall graph. While queries over most of the properties and relations
in the MAG are possible, I decided to focus on returning edges which
represent probable paths of communication between authors. To get this
large dataset into neo4j, I needed to convert the relations from the raw
tab-separated CSV files (TSV) to CSVs, then use the offline importer to
neo4j. To do this, I built a small conversion program in the programming
language Rust (\texttt{mag-csv} in source code) which could perform the
conversions quickly. I then fed the output of this conversion program
into neo4j's offline bulk import tool (\texttt{neo4j-import}) which
efficiently imported the data. I took this route because importing via
traditional queries is far too slow to be feasible for a dataset of this
size (the ``fast'' import still took several days on a laptop with a
relatively fast solid-state storage drive).

\hypertarget{the-authorcites-relation}{%
\subsection{\texorpdfstring{The \texttt{AuthorCites}
Relation}{The AuthorCites Relation}}\label{the-authorcites-relation}}

Once the data was in query-able form, I turned my focus to determining
the precise relations I would use to capture likely communication
relations between authors. To do this, I wanted to leverage citations
because they are instances where we have clear evidence in the MAG that
one work has influenced another. However, citations relate papers, not
authors, so the cited relation must be lifted to apply to authors.
Consider two authors \(A_1\) and \(A_2\). We say that \(A_1\) cites
\(A_2\) if and only if there exist papers \(P_1\) and \(P_2\) such that
\(A_1\) authored \(P_1\), \(P_1\) cites \(P_2\) and \(A_2\) authored
\(P_2\). Formally:

\begin{align*}
\texttt{AuthorCites}(A_1, A_2) = \exists_{ P_1, P_2 \in \texttt{Papers} } (                &\texttt{AUTHORED\_BY} (P_1, A_1) \land \\
   &\texttt{REFERENCES}(P_1, P_2) \land \\
   &\texttt{AUTHORED\_BY}(P_2, A_2) )
\end{align*}

All that is required is that a single pair of papers \(P_1\) and \(P_2\)
exist to maintain the \texttt{AuthorCites} relation. From this relation,
we infer that results were communicated from \(A_2\) to \(A_1\) by way
of \(A_1\) reading or reacting to \(A_2\)'s work, which indicates both a
general awareness of the cited author and the results presented in the
paper. This may be a cursory awareness if the citing author glanced at
the cited paper before using the citation strategically, however, that
is still transmitted information which could alter the direction of the
author's research. This is no perfect relation because many types of
communication happen without a subsequent citation. Furthermore,
citations themselves might represent little to no information
transferred. However, I argue that this method does a much better job of
approximating scientific community structure than a cycle, wheel,
complete or any other idealized graph would by virtue of being derived
from real data about real interactions.

An important point of comparison here is the co-authorship relation
which is often used in bibliometric community analysis. This simply
connects two authors when they have co-authored a paper together. While
co-authorship is a strong signal that information has been transmitted,
I argue it is overly restrictive to fit well with Zollman's definition
of edges representing communication of results between two parties.
Furthermore, when there is a sole author, co-authorship would not
reflect any other community members that the sole author was influenced
by. This feels wrong, especially when the author cited other papers, so
I chose to go with the \texttt{AuthorCites} relation.

While the \texttt{AuthorCites} relation does capture what we want at a
high level, it does not help limit the overall data processed in a given
query. To do this, I create the \texttt{AuthorCitesInField} which limits
connections to only those in which \(P_1\) and \(P_2\) are both in the
field of study of interest, \(F \subseteq \text{Papers}\). Thus, the
definition becomes:

\begin{align*}
\texttt{AuthorCites}(A_1, A_2) = \exists_{ P_1, P_2 \in F } (                &\texttt{AUTHORED\_BY} (P_1, A_1) \land \\
   &\texttt{REFERENCES}(P_1, P_2) \land \\
   &\texttt{AUTHORED\_BY}(P_2, A_2) )
\end{align*}

Because we are focused on specific communities, this definition works
well while significantly reducing query complexity by only quantifying
over every paper within a single field, rather than over every last
paper in the MAG. This does ignore interdisciplinary work, which is an
unfortunate downside to this approach, however, the decrease in query
complexity achieved here is what makes these queries feasible at all
using a relatively small machine.

Given the definition of \texttt{AuthorCitesInField}, I crafted the
following neo4j query. The query takes on a different form than the
definition to better leverage the graph structure and relations that
exist in the graph. There is no relation which links authors to fields
of study, so it is much faster in practice to enumerate a list of papers
in a field than a list of authors. This query, as written, will result
in duplicate relationships between authors, however, those are more
effectively de-duplicated in a graph processing library rather than in
the database.

\begin{verbatim}
MATCH (p1:Paper)-[:IN_FIELD]->(parent:FieldsOfStudy{id: $id})
WHERE p1.citationCount > 0
MATCH (p1:Paper)-[r:REFERENCES]->(p2:Paper)
MATCH (p2:Paper)-[:IN_FIELD]->(parent:FieldsOfStudy{id: $id})
WHERE p2 <> p1 AND p2.citationCount > 0
MATCH (p2:Paper)-[:AUTHORED_BY]->(a2:Author)
MATCH (p1:Paper)-[:AUTHORED_BY]->(a1:Author)
WHERE a1 <> a2
RETURN a1, a2
\end{verbatim}

\hypertarget{selecting-subgraphs}{%
\section{Selecting Subgraphs}\label{selecting-subgraphs}}

Now that I've described my method for relating authors via citations, I
tried the method out on several areas of research. To pick a diverse set
of fields of research, I used the MAG's fields of study as groupings and
its hierarchical organization of fields of study to select fields to
analyze. This organization places each field in a tree that descends
from a ``root'' field which has no parent.

I started by selecting ``social epistemology'' and ``peptic ulcer''. I
choose these two because Zollman publishes in the field of social
epistemology and writes about the field of peptic ulcer research in his
case study. After selecting these, I chose ``brain morphometry'' as a
subfield of neuroscience, ``monetary policy'' as a subfield of
economics, ``abstract algebra'' as a subfield of math and ``phonetics''
as a subfield of linguistics. My goal with this selection of fields is
to find fields of different sizes and with likely somewhat different
citation practices and cultures to ensure that my tests of Zollman's
model do not result in field-specific results.

In Tables \ref{tbl:networknec} and \ref{tbl:networkscc}, I detail the
resulting \texttt{AuthorCitesInField} graphs in relation to
co-authorship graphs, which serve as a point of comparison. I report
node and edge counts to give a general idea of the size of a field in
Table \ref{tbl:networknec}, where nodes are authors and edges are
\texttt{AuthorCitesInField} relations which connect authors to author's
they've cited via directed edges. Thus, the more nodes, the bigger the
field and the more edges, the more well connected the field. I report
basic stats pertaining to the connected components (both strong and
weak) in Table \ref{tbl:networkscc}. This includes both a count of the
total number of separate components as well as the size of the largest
one. It is important to note that because co-authorship is undirected,
strong and weak connectivity are the same for those networks.
Connectivity is a crude measure of the cohesion of a community which
partition the graph into sections such that no edge goes between
sections. These partitions of the graph represent fractures in the
community such that authors in separated components have never cited
anyone in any of the other sections. Having many large components
suggests either that fields are themselves fractured or that the MAG
mislabeled some author's work, leading them to be an island in a
spuriously assigned field. For larger, broader fields, it makes sense
that the field would be partitioned as it seems more likely that
research might be labeled under that field without much connection to
the primary body of work for the field.

A striking characteristic of most of the \texttt{AuthorCitesInField} and
co-authorship fields is the emergence of singular large weakly connected
components that contain most of the nodes in many, but not all, of the
selected fields. However, in nearly all cases, the weakly connected
component is much larger than the strongly connected component. This
effect is commonly observed in large graphs and most famously first
observed in the analysis of the early web
\autocite{broderGraphStructureWeb2000b}, where the large weakly
connected component did not imply a large strongly connected component.
These results emphasize the structural differences between directed and
undirected graphs, in that simply allowing the
\texttt{AuthorCitesInField} to be bidirectional can unify a fractured
field. Because Zollman's models are only meant to be run on connected
graphs, the model will, in practice, be run on connected subgraphs
rather than the overall fields.

\begin{longtable}[]{@{}lrr@{}}
\caption{Selected fields and resulting \texttt{AuthorCitesInField}
networks and Co-Authorship and associated node and edge counts.
\label{tbl:networknec}}\tabularnewline
\toprule
network & nodes & edges\tabularnewline
\midrule
\endfirsthead
\toprule
network & nodes & edges\tabularnewline
\midrule
\endhead
social epistemology AuthorCited & 651 & 1490\tabularnewline
social epistemology CoAuthor & 1663 & 3777\tabularnewline
brain morphometry AuthorCited & 5154 & 77371\tabularnewline
brain morphometry CoAuthor & 8584 & 139059\tabularnewline
monetary policy AuthorCited & 17620 & 454030\tabularnewline
monetary policy CoAuthor & 30418 & 108882\tabularnewline
abstract algebra AuthorCited & 311 & 968\tabularnewline
abstract algebra CoAuthor & 977 & 2400\tabularnewline
peptic ulcer AuthorCited & 20970 & 562167\tabularnewline
peptic ulcer CoAuthor & 49017 & 317141\tabularnewline
phonetics AuthorCited & 5766 & 58192\tabularnewline
phonetics CoAuthor & 10399 & 36707\tabularnewline
\bottomrule
\end{longtable}

\begin{longtable}[]{@{}lrrrr@{}}
\caption{Selected fields and resulting \texttt{AuthorCitesInField} (AC)
networks and Co-Authorship (CA) and associated strongly connected
component (SCC) counts, weakly connected component (WCC) counts, the
size of the largest strongly connected component, and the size of the
largest weakly connected component.
\label{tbl:networkscc}}\tabularnewline
\toprule
network & largest SCC & num SCCs & largest WCC & num WCCs\tabularnewline
\midrule
\endfirsthead
\toprule
network & largest SCC & num SCCs & largest WCC & num WCCs\tabularnewline
\midrule
\endhead
social epistemology AC & 10 & 615 & 560 & 30\tabularnewline
social epistemology CA & 82 & 596 & 82 & 596\tabularnewline
brain morphometry AC & 1661 & 3392 & 5049 & 11\tabularnewline
brain morphometry CA & 5186 & 576 & 5186 & 576\tabularnewline
monetary policy AC & 9616 & 7913 & 17541 & 32\tabularnewline
monetary policy CA & 13306 & 5974 & 13306 & 5974\tabularnewline
abstract algebra AC & 12 & 285 & 192 & 29\tabularnewline
abstract algebra CA & 25 & 337 & 25 & 337\tabularnewline
peptic ulcer AC & 8383 & 12444 & 20449 & 67\tabularnewline
peptic ulcer CA & 14698 & 7371 & 14698 & 7371\tabularnewline
phonetics AC & 2049 & 3640 & 5608 & 29\tabularnewline
phonetics CA & 3734 & 2064 & 3734 & 2064\tabularnewline
\bottomrule
\end{longtable}
