\hypertarget{zollman-models}{%
\chapter{Zollman Models}\label{zollman-models}}

Given my account of experimental mechanistic simulations, we can now
dive deeper into Zollman's network epistemology project. I'll pay
special attention to how his modeling begins from a mathematical model,
uses simulation to add complexity and mimic mechanism which can be
manipulated. This section will be divided into three distinct parts.
I'll start by determining what Zollman seeks to achieve by modeling in
\emph{The Epistemic Benefit of Transient Diversity}
\autocite{zollmanEpistemicBenefitTransient2009}, then I'll dissect the
model structure and conclude by pointing forward to my extensions to his
work.

\hypertarget{zollmans-project}{%
\section{Zollman's Project}\label{zollmans-project}}

Before describing the model, it is important to understand Zollman's
project and research purpose so as to have a good characterization of
his intent with this model. As discussed in the previous section, the
intent of the model matters as it determines which target the model is
supposed to represent. The choice of target and the model's fit with the
goals of the project can make or break a model. Thus, here I discuss a
charitable interpretation of Zollman's intentions and use those
intentions to more rigorously specify his target world.

At a high level, Zollman aims to understand how a diverse cognitive
division of labor in science develops. Zollman notes that science has a
striking property that different scientists work on very different
problems and even those working on the same problem pursue radically
different strategies. Zollman paints this diversity as
counter-intuitive, citing Kuhn as saying that any ``shared algorithm''
for deciding what to work on would lead to a lack of disagreement and to
all scientists working on the same thing
\autocite[p.~332][]{kuhnCollectiveBeliefScientific1977}. He also cites
Philip Kitcher \autocite{kitcherDivisionCognitiveLabor1990a} and Michael
Strevens \autocite{strevensRolePriorityRule2003a} as taking a related
stance which explains diversity as resulting from reward structures, for
example: one which disproportionately rewards the first discoverer.

Zollman seeks to reframe the problem as a social epistemic problem of a
group of individuals trying to collectively discover which theory will
be most fruitful to work on. To model this, Zollman assigns a
ground-truth probability of success to any given research direction. For
example, an action \(A\) might have a 50\% probability of panning out
successfully if a researcher chooses to work on it. This probability, to
Zollman, is best understood as the probability of choosing a research
direction that leads to true, useful knowledge given a certain amount of
research effort. For example, research into CRISPR sequences and the
cas9 protein turned out to be a hugely productive research endeavor,
which hints that that line of research had a higher probability of
success than the average research project. Conversely, cold fusion
research did not pan out, which might have been because that line of
research inherently had a low probability of success given the goal of
room-temperature fusion seems physically implausible now. We wouldn't
expect a 100\% probability of success for any endeavor as bad luck, lack
of funding, and other factors can sink even the best of projects.

There are many ways to interpret this probability of success as Zollman
does not offer a hard and fast interpretation in his own work. However,
this conception of some degree of ``hardness'' of a line of research
relative to reward captured in the above description is more or less
sufficient for understanding the high-level division of labor as Zollman
seeks to do.

Zollman first makes a case for the value of diversity by highlighting
the story of peptic ulcer disease. Zollman exhibits this line of
research as an example of scientists collectively choosing the wrong,
lower probability of success line of research to the detriment of the
field. A scientist in 1954 published a highly influential study which
claimed to rule out bacteria as the cause of the disease, leading most
scientists to work on theories which assumed excess acid as the cause.
This incorrect assumption led to failed treatments for over 50 years
until a scientist revived the bacterial theory by ingesting the
bacteria, causing the disease in himself, then using antibiotics to cure
himself. While this scientist did eventually right the course of that
field, the anecdote begs the question: why did the field get so off
course and how could that have been prevented? Zollman posits that the
field jumped too quickly to monoculture and designs his model to
determine which practices might encourage a healthy level of diversity.

Within this framing, Zollman seeks to see if the following rules lead to
diversity in the field:

\begin{enumerate}
\def\labelenumi{\arabic{enumi}.}
\tightlist
\item
  Limiting agents to working on a single theory at a time
\item
  Giving agents prior beliefs about each theory
\item
  Allowing agents to observe limited information from others in the
  community
\end{enumerate}

Zollman ultimately finds that these rules do encourage diversity when
information is sufficiently limited or when agents have extreme priors.
Though, when both cases are true, the diversity becomes detrimental and
scientists never drop inferior theories. From these results, Zollman
concludes that diversity is not an inherent goal, as it can prevent
convergence to a ``better'' theory, though it is helpful temporarily to
reach an optimal result. Furthermore, Zollman emphasizes that this
social model demonstrates that behavior that seems sub-optimal for an
individual can become optimal within a community structure. From these
modeling results, Zollman concludes that limiting information exposed to
scientists or scientists holding more extreme priors creates transient
diversity, which ensures theories aren't discarded too quickly so the
overall community reaches more optimal results.

Given these intentions, I attribute Zollman's transient diversity model
(referred to as \(M_d\)), as corresponding to the target world which
contains real scientists acting within a real community structure
(\(W_s\)). I argue a correspondence to the real world is a charitable
interpretation as Zollman includes a real-world anecdote about real
scientists and concludes by calling transient diversity a virtue for
science. Furthermore, Zollman claims that the peptic ulcer disease snafu
might not have been so damaging ``had Palmer's result not been
communicated so widely or had people been sufficiently extreme in their
beliefs that many remained unconvinced by his study''
\autocite[p.~33][]{zollmanEpistemicBenefitTransient2009}, indicating
that he does take these findings as applying to actual scientists.

However, there is one important distinction to be made with the model
between the actual and hypothetical worlds. A model concerning an actual
world would model the world and phenomena we could, in principle
observe, whereas one targeting a hypothetical world models phenomena we
couldn't directly observe without making some sort of modification or
intervention. Thus, he needs a manipulable model to convincingly
establish an explanation through intervention. If the proposed
interventions seem to cause the phenomena of interest, then they seem
like plausible interventions. Having a simulation as a model is critical
to this because it allows Zollman to vary aspects of it and show that
phenomena appear and disappear in response.

\hypertarget{the-model-algorithmically}{%
\section{The Model Algorithmically}\label{the-model-algorithmically}}

Before discussing model-target relationships, I'll first partition the
model into pieces that each have their own distinct targets as described
in the previous chapter. Zollman pulls the generic model structure from
an earlier work by Bala and Goyal \autocite{balaLearningNeighbours1998}
which presents a model that combines graph structure and bandit problems
as a means of modeling social learning. Thus, to understand Zollman's
models, we first must understand Bala and Goyal's model \(M_{bg}\).
While such models are inherently about social structure, I'll start by
describing individual behavior to emphasize how social structure affects
this behavior.

In \(M_{bg}\), we refer to the piece of model machinery meant to
represent a person as an \emph{individual}. Each individual is tasked
with picking a way to act without knowing \emph{a priori} the
probabilities of success for each possible action. Furthermore, the
individuals are arranged in a graph structure that defines the neighbors
of each individual. More formally, I can define the structure of the
model as follows:

\[ M_{bg} = (G, I, A) G = (V, E) I = \{ i_1, \ldots, i_{|V|} \} A = \{ a_1, \ldots, a_n \}\]

In the above, \(G\) is a graph, which is composed of a set of vertices
\(V\) and a set of edges \(E\). \(I\) is a list of individuals, each
corresponding to a vertex in the graph \(G\). Finally, \(A\) holds all
the possible actions to take in the world.

Now, consider an individual \(i_n\). \(i_n\) corresponds to the node
\(n \in V\) and holds beliefs about each of the actions in \(A\). An
individual might hold a set of beliefs \(B\) where \(|B| = |A|\) and
each element \(b_j\) is a belief distribution about \(a_j\). In Zollman,
each belief is modeled as a Beta distribution that is randomly
initialized with values \(\alpha, \beta \in \left[0, 4\right]\):

\[b_j \sim Beta(\alpha, \beta)\]

Each action is modeled as a binomial distribution which, in turn, models
some number of trials with a given probability of success (Zollman uses
\(n = 1000\) and \(p = 0.5\) and \(p = 0.499\) for the two possible
actions in his models):

\[a_k \sim B\left(n = 1000, p \in \left\{0.5, 0.499 \right\}\right)\]

Each individual is composed of a set of beliefs and mechanisms for
running experiments, sharing data with neighboring individuals, and
updating their beliefs in response to observed data. At each step, each
individual picks the highest probability belief \(b_j\), draws from the
corresponding action \(a_j\) and receives a number of successes \(s\)
out of the \(n = 1000\) trials. The individual then shares its results
with neighbors and compiles the total number of successes \(s_j\) and
trials \(n_j\) for each action \(a_j\). Then for each action \(a_j\),
the individual updates its corresponding beliefs:

\[a_j = \beta(\alpha + s_j, \beta + n_j - s_j)\]

This process continues for a set number of steps. Once it is complete,
each agent can be assessed by determining if its beliefs instruct it to
pick the action defined to have the highest probability.

\hypertarget{next-steps}{%
\section{Next Steps}\label{next-steps}}

Zollman's graph structures are fairly unrealistic and thus it isn't
clear those structures actually represent the target, which is real
scientists in real communities. While I'll go into the exact structures
he used later, this section specifically seemed ripe for a more
realistic modeling method due to the proliferation of detailed empirical
data on scientific community structure.

There are many other parts of this model that are very simplified and
which I did not re-implement with a more accurate model. Modeling people
as only having two choices about what to do and limiting people to only
hold a beta distribution's worth of information about both of them seems
very oversimplified. Modeling researcher's opinions of an entire
research direction as a single beta distribution approximating
probability of success feels reductionist when all sorts of factors from
interest to cultural fit play a large role in what fields researchers
decide to spend years of their lives working on. However, increasing the
resolution of the community structures adds realism where Zollman is
most interested. Zollman is interested in what community structures lead
to better science, so the community structures he discusses should be as
realistic as possible. Even if he is going for a normative claim, he
should be arguing for something that seems realistic and possible and
the best starting point for that is current scientific structures.
